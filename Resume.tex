% LaTeX source of my resume
% =========================

% Original from https://github.com/cies/resume

% See the `README.md` file for more info.

% Start a document with the here given default font size and paper size.
\documentclass[11pt,a4paper]{article}

% Set the page margins.
\usepackage[a4paper,margin=0.75in]{geometry}

% Setup the language.
\usepackage[english]{babel}
\hyphenation{Some-long-word}

% Makes resume-specific commands available.
\usepackage{resume}

% Indents the first paragraph
\usepackage{indentfirst}

% Removes indentation for in a list environment.
\usepackage{enumitem}

\begin{document}  % begin the content of the document
\sloppy  % this to relax whitespacing in favour of straight margins


% title on top of the document
\maintitle{Christine J. Tang}{July 4, 1998}{Last update on \today}

\nobreakvspace{0.3em}  % add some page break averse vertical spacing

% \noindent prevents paragraph's first lines from indenting
% \mbox is used to obfuscate the email address
% \sbull is a spaced bullet
% \href well..
% \\ breaks the line into a new paragraph
\noindent\href{mailto:chrisjtang98.at.gmail.dot.com}{chrisjtang98\mbox{}@\mbox{}gmail.com}\sbull
\textsmaller(973) 960-2295\sbull
\href{https://github.com/chris-j-tang}{github.com/chris-j-tang}\sbull
\href{https://www.linkedin.com/in/chris-j-tang/}{linkedin.com/in/chris-j-tang}

\spacedhrule{0.9em}{-0.4em}  % a horizontal line with some vertical spacing before and after

\roottitle{Summary}  % a root section title

\vspace{-1.8em}  % some vertical spacing
% \begin{multicols}{2}  % open a multicolumn environment
{\bodytext \noindent \emph{As a self-motivated individual who pursues quality over quantity in a fast-paced world and prioritizes reducing tech debt over shipping buggy code, I enjoy leading and being a part of a team working on cutting-edge innovations. }}
% \\
% \\
% Biography
% \end{multicols}

\vspace{0.5em}
\spacedhrule{0em}{-0.4em}

\roottitle{Education}

\headedsection
  {\href{http://rpi.edu/}{Rensselaer Polytechnic Institute}}
  {\textsc{Troy, NY}} {%
  \headedsubsection
    {Computer Science, Math (BSc) -- \textnormal{~Economics Minor} -- 3.8/4.0}
    {May 2019} {\\}
}

\spacedhrule{0em}{-0.4em}

\roottitle{Experience}

\headedsection
  {\href{http://www.hass.rpi.edu/pl/hass-research-facilities/?objectID=10049}{Language Endowed Intelligent Agents Laboratory \acr{(LEIA)}}}
    {\textsc{Troy, NY}} {%
    \headedsubsection
      {Research Assistant \& System Administrator}
      {Feb \apo17 -- present}
      {\bodytext{
        \emph{Ubuntu Linux, Bash shell/scripts, Python, git, svn}
        \renewcommand\labelitemi{{\boldmath$\cdot$}}
        \begin{itemize}[leftmargin=*] % Decrease space before indent to reduce white space
          \item Implemented Ubuntu Linux server data migration remotely using Bash shell to better distribute server payload, increase server security, and expand storage space
          \item Worked with senior sysadmins to reimage corrupted central server from backups 
          \item Configured and wrote bash scripts to automate user account maintenace and to restart laboratory tools and scripts on server restarts
        \end{itemize}
      }}
  }
  
\headedsection
  {\href{https://github.com/HighSchoolHacking/GLS}{General Language Syntax \acr{(GLS) -- Microsoft/RCOS Project}}}
    {\headedsubsection
        {Collaborator}
        {Feb \apo17 -- present}
        {\bodytext{
          \emph{Typescript, C\#, Python, git}
          \renewcommand\labelitemi{{\boldmath$\cdot$}}
          \begin{itemize}[leftmargin=*]
            \item Worked with a Microsoft \acr{FTE} to streamline the core command rendering engine driver 
            \item Designed and implemented prevention of banned keywords for parameter variable names
            \item Increased functionality to GLS libraries for conversions to various object-oriented languages.
          \end{itemize}
        }}
    }

% \headedsection
%   {\href{http://coling2018.org}{27th International Conference on Computational Linguistics \acr{(COLING 2018)}}}
%   {\textsc{Santa Fe, NM}}
%   {\headedsubsection
%     {Webmaster}
%     {April \apo17 -- present}
%     {\bodytext{Filler}}
%   }

\headedsection
  {\href{https://en.wikipedia.org/wiki/Wikipedia:Getting_to_Philosophy}{Wikipedia Philosophy Crawler}}
  {\headedsubsection
    {Personal Project}
    {Winter \apo16}
    {\bodytext{
      \emph{Python, Beautiful Soup 4}
      \renewcommand\labelitemi{{\boldmath$\cdot$}}
      \begin{itemize}[leftmargin=*]
        \item Designed and completed a Python script using a HTML parser to automate crawling through Wikipedia articles until the \href{https://en.wikipedia.org/wiki/Philosophy}{Philosophy} page is reached.
        \item Scraped pages for hyperlinks, and used regex to remove those italicized or in parentheses
        \item Implemented features such as avoiding possible repeats by storing visit history in Python dictionary and including the option of specifying a starting page or start randomly
      \end{itemize}
      % Made it a personal challenge to use {\bf{Python}} to access and pull information from the Internet. Around this time, I was introduced to the Wikipedia Philosophy phenomenom: clicking the first
      % non-italicized hyperlink not in parentheses on a page and all subsequent ones, will ultimately yield the \href{https://en.wikipedia.org/wiki/Philosophy}{Philosophy} page. Confronted by the difficulty of 
      % not having written Python script before, I soon found out that entire libraries existed for people in my predicament. I ended up using the \href{https://www.crummy.com/software/BeautifulSoup/}{\bf{Beautiful Soup \acr{HTML}}} parser library, which took care
      % of extracting the Wikipedia page's source code for me to parse. I used regular expressions to remove hyperlinks italicized and enclosed in parentheses, and continued the search on the first valid hyperlink. I stored the past history in a hash table to prevent
      % loops, and output the current page the crawler was on for the user's viewing pleasure.
    }}
  }

\headedsection
{\href{https://www.vexrobotics.com/}{VEX Robotics}}
{\headedsubsection
  {Team Leader \& Member}
  {Summer \apo15 -- Fall \apo15}
  {\bodytext{
    \emph{RobotC, Arduino IDE, git}
    \renewcommand\labelitemi{{\boldmath$\cdot$}}
    \begin{itemize}
      \item Led a team of 3 to build and code an autonomous robot to navigate through corridors using magnetic and proximity sensors
      \item Led a team of 5 to construct and maintain a controlled-robot to play a version of basketball
      \item Designed the underlying code base for the VEX competition with a team of 3
    \end{itemize}
  }}
}

\spacedhrule{0.5em}{-0.4em}

% \roottitle{Skills}

% \inlineheadsection  % special section that has an inline header with a 'hanging' paragraph
%   {Technical expertise:}
%   {Most Proficient: C++, Python}
%   {Proficient: }
  % {Leading and recruiting teams of software engineers.  Big fan of Agile methodologies, continuous delivery and functional programming.  Enjoys writing Ruby/\nsp Python/\nsp Java/\nsp \CPP~and Haskell.  Solid knowledge of the full web technology stack.  Able to architect \textit{and} implement distributed/\acr{HA} systems.  Strong Linux administration skills (e.g.\ Bash scripting, Apache/\acr{NGINX}, Postgres/My/No\acr{SQL}, ElasticSearch).  Well experienced with virtualization/containerization (Docker/Kubernetes, \acr{KVM}, Xen and several \acr{AWS} solutions) and DevOps (Puppet).  Emacs user.}

% \vspace{0.5em}
% \inlineheadsection
%   {Natural languages:}
%   {English \emph{(fluent)}, Chinese \emph{(fluent)}, French \emph{(elementary proficiency)}.}


% \spacedhrule{1.6em}{-0.4em}

\roottitle{Honors \& Awards}
  \indent Dean's Honor List: Fall 2016 \& Spring 2017 \\
  \indent Rensselaer Leadership Award \\      
  \indent Rensselaer Recognition Award

\spacedhrule{1.6em}{-0.4em}

\roottitle{Interests}

\inlineheadsection
  {Non-exhaustive and in alphabetical order:}
  { bouldering, chess, running, violin, wargames \acr{(CTF)} }
\end{document}

